\documentclass{resume} % Use the custom resume.cls style

\usepackage[left=0.4 in,top=0.4in,right=0.4 in,bottom=0.4in]{geometry} % Document margins
\newcommand{\tab}[1]{\hspace{.2667\textwidth}\rlap{#1}} 
\newcommand{\itab}[1]{\hspace{0em}\rlap{#1}}
\name{Rishi Kumar} % Your name
% You can merge both of these into a single line, if you do not have a website.
\address{\href{mailto:rsi.dev17@gmail.com}{rsi.dev17@gmail.com} \\ West Bengal, IN} 
\address{\href{https://www.rishikumar.dev}{rishikumar.dev} \\ \href{https://www.linkedin.com/in/rsidev}{linkedin.com/in/rsidev} \\ \href{https://www.github.com/k3yss}{github.com/k3yss}}  %

\begin{document}
	
	%----------------------------------------------------------------------------------------
	%	OBJECTIVE
	%	\begin{rSection}{OBJECTIVE}

	%	{Software Engineer with 2+ years of experience in XXX, seeking full-time XXX roles.}


	%	\end{rSection}
	%----------------------------------------------------------------------------------------
	
	%----------------------------------------------------------------------------------------
	%	EDUCATION SECTION
	%----------------------------------------------------------------------------------------
	
	\begin{rSection}{Education}
		
		{\bf Indian Institute of Information Technology}  \hfill {Kalyani, West Bengal}\\
		\textit{B.Tech Compter Science \hfill {Dec 2021 - June 2025}}\\
		\textbf{Secretary} Free and Open Source Club, Freescape 
		
		{\bf Chinmaya Vidyalaya, Bokaro}  \hfill {Bokaro, Jharkhand}\\
		\textit{Senior Secondary School \hfill {2018 - 2020}} \\
		%Minor in Linguistics \smallskip \\
		Received a \textbf{Gold Medal} for scoring the highest marks in \textbf{Computer Science}
		%Member of Upsilon Pi Epsilon \\
		
	\end{rSection}
	%----------------------------------------------------------------------------------------
	%	WORK EXPERIENCE SECTION
	%----------------------------------------------------------------------------------------

	\begin{rSection}{WORK EXPERIENCE}
		
		\textbf{Google Summer Of Code 2023} \hfill Remote\\
		\textit{Contributor to Tokodon} \hfill \textit{May 2023 - September 2023}
		\begin{itemize}
			\itemsep -3pt {} 
			\item Implemented \textbf{Moderation Tools} using \textbf{Mastodon API}. Used \textbf{QT/C++} on the backend and \textbf{QML} for user interface.
			\item Contributed to \textbf{Mastodon's GitHub} addressing several \textbf{Issues} and \textbf{Merged Requests} addressing several bugs.
			\item Contributed \textbf{6k+ lines} of code over several Merged Requests and detailed the process over several public blog posts. 
		\end{itemize}
		
		\textbf{KDE} \hfill Remote\\
		\textit{Maintainer, Contributor} \hfill \textit{Dec 2022 - Present}\\
		Started contributing as part of \textbf{Season of KDE 2023} where I worked on porting Tokodon a KDE's Mastodon Client to work offline after which I wrote automated Appium tests and integrated them in GitLab CI. 
		
	\end{rSection} 
	
	%----------------------------------------------------------------------------------------
	% TECHINICAL STRENGTHS	
	%----------------------------------------------------------------------------------------
	\begin{rSection}{SKILLS}
		
		\begin{tabular}{ @{} >{\bfseries}l @{\hspace{6ex}} l }
			Programming Languages: & C/C++, Javascript, Python, Go, Rust\\
			Frameworks: & Qt(QtQuick, QML), React, NextJS \\
			Tools: & Git, Linux, Docker, Github Actions\\
		\end{tabular}\\
	\end{rSection}
	
	%----------------------------------------------------------------------------------------
	%	PROJECTS SECTION
	%----------------------------------------------------------------------------------------

	\begin{rSection}{PROJECTS}
		\vspace{-1.25em}
		\item \textbf{Trinity} \textit{python, tensorflow, pyside6, matlab} \hfill {\href{https://x.rishikumar.dev/trinity}{x.rishikumar.dev/trinity}}\\
			{Trinity \textbf{identifies and extracts FEC schemes} from unknown demodulated signals using an \textbf{improved RNN model} trained on 1 million+ samples. Trinity can be used as a Command Line and Graphical User Interface which is written in python using pyside6.}
		\item \textbf{Titan} \textit{golang} \hfill {\href{https://x.rishikumar.dev/titan}{x.rishikumar.dev/titan}}\\
			{A WIP \textbf{compiler} for my implementation of \textbf{lox programming language} using \textbf{recursive descent technique}  for parsing.}
		\item \textbf{Gofetch} \textit{golang} \hfill {\href{https://x.rishikumar.dev/gofetch}{x.rishikumar.dev/gofetch}}\\
			{A \textbf{system fetch} utility written in golang.}
	\end{rSection} 


	
	%----------------------------------------------------------------------------------------
	\begin{rSection}{Accomplishments} 
		\textbf{E-Yantra} {Robotics Competition} \hfill IIT Bombay\\
		{Our team were the top 10-30 ranked team in the competition and managed to reach the semi-final stage out of 1000+ teams.}\\	
		\textbf{Smart India Hackathon} {Engineering Competition} \hfill Indian Government\\
		{Led my team to the Grand Finale where we worked on problem statement SIH1447 by the National
Technical Research Organization for identifying and extracting FEC schemes from unknown demodulated signals.} 	
	\end{rSection}
	%----------------------------------------------------------------------------------------
\end{document}
